This file is a designed as a single test that incorporates most of the
\emph{commonly} used MultiMarkdown features. This allows a single test file to
identify common structures that are not supported yet -- particularly useful
when developing a new output format.

\part{Basic Blocks }
\label{basicblocks}

paragraph

\begin{itemize}
\item list

\item items

\end{itemize}

\begin{verbatim}
fenced
code
\end{verbatim}

\begin{verbatim}
indented
code
\end{verbatim}

\begin{quote}
blockquote
\end{quote}

\texttt{code span}

\emph{emph} and \textbf{strong} and \textbf{\emph{both}}

\emph{emph} and \textbf{strong} and \textbf{\emph{both}}

\chapter{Escapes }
\label{escaped}

\begin{enumerate}
\item \$

\item \#

\item [

\end{enumerate}

\part{Footnotes }
\label{footnotes}

Foo.\footnote{This is an inline footnote}

Bar.\footnote{And a reference footnote.}

Cite.\citesyntax TBD

\part{Links and Images }
\label{linksandimages}

\href{http://foo.net/}{link}\footnote{\href{http://foo.net/}{http:/\slash foo.net\slash }} and \href{http://bar.net}{link}\footnote{\href{http://bar.net}{http:/\slash bar.net}}

\begin{figure}[htbp]
\centering
\includegraphics[keepaspectratio]{http://foo.bar/}
\caption{test}
\end{figure}

\autoref{math}

foo (\autoref{math})

\autoref{bar}

\autoref{bar}

\part{Math }
\label{math}

foo \({e}^{i\pi }+1=0\) bar

\[ {x}_{1,2}=\frac{-b\pm \sqrt{{b}^{2}-4ac}}{2a} \]

foo ${e}^{i\pi }+1=0$ bar

foo ${e}^{i\pi }+1=0$, bar

$${x}_{1,2}=\frac{-b\pm \sqrt{{b}^{2}-4ac}}{2a}$$

\part{Smart Quotes }
\label{smartquotes}

``foo'' and `bar' -- with --- dashes{\ldots}

\part{CriticMarkup }
\label{criticmarkup}

\underline{foo}

\sout{bar}

\sout{foo}\underline{bar}

\todo{foo}

\hl{bar}

\part{Definition Lists }
\label{definitionlists}

\begin{description}
\item[foo]

\item[bar]

*foo bar

baz bat*
\end{description}

\part{Horizontal Rules }
\label{horizontalrules}

\begin{center}\rule{3in}{0.4pt}\end{center}
